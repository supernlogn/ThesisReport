\chapter{Εισαγωγή}
Η Τεχνητή Νοημοσύνη είναι ο κλάδος/τομέας της επιστήμης της πληροφορικής, που ασχολείται με την σχεδίαση και κατασκευή ευφυών συστημάτων, δηλαδή συστημάτων που διαθέτουν χαρακτηριστικά που σχετίζονται με την ανθρώπινη νοημοσύνη και συμπεριφορά. Ο πιο ισχυρός υπολογιστής που γνωρίζει ο άνθρωπος μέχρι στιγμής είναι ο ίδιος ο εγκέφαλός του(όπως βέβαια και οι εγκέφαλοι άλλων ζώων). Αυτός για να φτάσει στο σημείο να χαίρει της τωρινής υπολογιστικής του ικανότητας χρειάστηκε εκατομμύρια χρόνια βιολογικής εξέλιξης. Ακόμα και με αυτή την υπολογιστική ικανότητα που διαθέτει συνεχίζει να χρησιμοποιεί τη τεχνική μετάδοσης γνώσης, παρά τα χρόνια εξέλιξής του. Αυτό σύμφωνα με τη δαρβινική θεωρία \cite{74} τη φυσικής επιλογής, δείχνει πως ήταν μία βελτιστοποίηση στη λειτουργία του (δεν ισχυρίζεται η τελειότητα αυτού) η οποία δοκιμάστηκε και επέζησε μέχρι στιγμής.  

Ως εκ τούτου, ο σχεδιασμός αλγορίθμων και μοντέλων που εκπαιδεύονται μέσω της μετάδοσης γνώσης είναι μία αρκετά δοκιμασμένη επιλογή και μία αρκετά δικαιολογημένη απόφαση. Ωστόσο, έχοντας πάλι ως παράδειγμα τον εγκέφαλο, τα νευρωνικά δίκτυα εντοπισμού αντικειμένων θα έπρεπε να απαιτούν λιγότερες παραμέτρους και να επεξεργάζονται την είσοδό τους σε πραγματικό χρόνο (τουλάχιστον 24 καρέ /δευτερόλεπτο). Η τελευταία αυτή απαίτηση σε συνδυασμό με την προηγούμενη οδηγούν στη διαμόρφωση του προβλήματος που περιγράφεται παρακάτω.

Επιπλέον, η μετάδοση γνώσης είναι ένα από τα εργαλεία που είναι στενά συνδεδεμένο με την έννοια της Γενικής Τεχνητής νοημοσύνης(Artificial General Intelligence). Αυτό συμβαίνει γιατί μοντελοποιείται ένα πολύ σημαντικό κομμάτι της διαδικασίας του εγκεφάλου: η ίδια η μάθηση.

Ένας σημαντικός τομέας που ωθεί τα σύγχρονα επιτεύγματα στη τεχνητή νοημοσύνη είναι η εμφάνιση και εξέλιξη του κλάδου της Βαθιάς Μηχανικής Μάθησης (Deep learning - DL).Η χρήση τεχνικών βαθιάς μάθησης στην επίλυση προβλημάτων Μηχανικής Όρασης, έχει κατορθώσει να αντιμετωπίσει περίπλοκα προβλήματα τα οποία μέχρι και πριν από λίγα χρόνια θεωρείτο ακατόρθωτο να λυθούν. Ακόμα και η χρήση της DL σε ενσωματωμένα συστήματα με μικρή μνήμη και απαιτήσεις ταχύτατης επεξεργασίας δίνει καλύτερα αποτελέσματα από ότι άλλοι αλγόριθμοι.

Σήμερα, το γενικότερο πρόβλημα της ταυτόχρονης αναγνώρισης και εντοπισμού αντικειμένων σε εικόνες χρησιμοποιεί εκτεταμένα Νευρωνικά Δίκτυα Συνέλιξης (Convolutional Neural Networks - CNNs). Η εκπαίδευση και η επανεκπαίδευση αυτών έχει επισημανθεί ότι ωφελείται αρκετά από τη χρήση τεχνικών μετάδοση γνώσης \cite{55}. Σε αυτό παίζουν ρόλο και άλλοι οικονομοτεχνικοί λόγοι όπως το μέγεθος των διαθέσιμων δεδομένων. Σε συνδυασμό όλων των παραπάνω η εργασία καλείται να δώσει κάποιες παρατηρήσεις προς την επίλυση του παρακάτω προβλήματος.

\section{Περιγραφή του Προβλήματος}
Παρόλο που τα νευρωνικά δίκτυα εντοπισμού αντικειμένων από εικόνες έχουν φτάσει στο σημείο να έχουν ικανοποιητική ακρίβεια και να εκτελούνται σε ενσωματωμένα συστήματα, δεν έχουν μελετηθεί όλες οι διαφορές τους και οι σχέσεις τους με αυτά που έχουν μεγάλο αριθμό παραμέτρων και εκτελούνται με μεγαλύτερη απαίτηση ισχύος. Η κύρια μελέτη αυτών των συστημάτων έγκειται στον χρόνο εκτέλεσης, την ακρίβεια και πρόσφατα τη μνήμη \cite{1} και την απαίτηση σε ισχύ. Ένα μεγάλο κεφάλαιο στην κατεύθυνση της έρευνας για τα νευρωνικά δίκτυα που μπορούν να εκτελούνται σε ενσωματωμένα συστήματα είναι η μετάδοση γνώσης.

Είναι σημαντικό για παράδειγμα ένα ρομπότ να μπορεί να χρησιμοποιήσει την προηγούμενα επεξεργασμένη (και μη) πληροφορία που διαθέτει για την ανάκτηση και την επεξεργασία καινούριας. Κατά αυτό τον τρόπο μειώνεται ο χρόνος εκπαίδευσής του σε μία καινούρια εργασία. Ωστόσο, είναι θεμιτό οι ενσωματωμένες συσκευές να είναι όσο πιο “ελκυστικές” γίνεται και συνήθως αρκετά μικρότερες σε μέγεθος από μία συστοιχία υπολογιστών \cite{70, 71}, ανάλογα με την εργασία που επιθυμούμε να εκτελέσουν. Αυτό, έχει ως αποτέλεσμα να μην μπορούν να τοποθετηθούν ογκώδη, άρα με μεγάλη επεξεργαστική ισχύ, υπολογιστικά συστήματα στα ενσωματωμένα συστήματα.

Ήδη η μετάδοση γνώσης χρησιμοποιείται συνεχώς σε προβλήματα ενισχυόμενης μάθησης(RL) \cite{72, 73}, ώστε να μπορεί το μοντέλο ενός πράκτορα να μεταφερθεί από την προσομοίωση στην πραγματικότητα. Το ίδιο απαιτείται και στην περίπτωση της τεχνητής όρασης με σκοπό την εξοικονόμηση δεδομένων, την ταχύτερη και την αποτελεσματικότερη εκπαίδευση. Το πρόβλημα είναι πως δεν έχουν γίνει πειραματισμοί και μελέτες για την καταγραφή της συμπεριφοράς των νευρωνικών δικτύων εντοπισμού αντικειμένων ενσωματωμένων συστημάτων ως προς τη μετάδοση γνώσης.

\section{Σκοπός-Συνεισφορά της Διπλωματικής Εργασίας}
Η παρούσα διπλωματική εργασία μελετά την επαναχρησιμοποίηση νευρωνικών δικτύων συνέλιξης (CNN) σε εφαρμογές ταυτόχρονης αναγνώρισης και εντοπισμού αντικειμένων (object recognition and localization - object detection) σε εικόνες.

Βασικός σκοπός είναι η μεταφορά πληροφορίας μοντέλων CNN από προηγούμενες εφαρμογές, διατηρώντας κατά το δυνατόν σταθερή την αρχιτεκτονική ώστε να επιτρέπεται συνεχώς η εφαρμογή τους σε προβλήματα πραγματικού χρόνου. Μία ακόμα απαίτηση είναι η ελάχιστη μνήμη του αλγορίθμου, προκειμένου να είναι δυνατή η εκτέλεσή του από όσο περισσότερες ενσωματωμένες συσκευές. Επίσης, ζητείται η κατά το δυνατόν υψηλότερη ακρίβεια του μοντέλου μετά από την εκπαίδευση του και μεταφορά γνώσης. Τέλος, απαιτείται η κατά το δυνατόν γρηγορότερη επανεκπαίδευσή του. Όλες αυτές οι απαιτήσεις εξετάζονται κατά το πόσο είναι εφικτές και τι περιορισμούς προϋποθέτουν. Η εξέταση αυτή γίνεται πρώτη φορά μέχρις στιγμής καθότι η μετάδοση γνώσης είναι αρκετά καινούρια τεχνική στο τομέα της τεχνητής νοημοσύνης όπως και τα νευρωνικά δίκτυα εντοπισμού πραγματικού χρόνου.

Επίσης, παρουσιάζεται ένας πιο γρήγορος αλγόριθμος για την εκπαίδευση του νευρωνικού δικτύου εντοπισμού \textit{SqueezeDet}, όπως και η εφαρμογή ενός νέου τρόπου βελτιστοποίησης υπερπαραμέτρων των νευρωνικών δικτύων. Τέλος, γίνεται και μία πειραματική αναφορά στην πλαστικότητα των νευρώνων των δικτύων εντοπισμού αντικειμένων για ενσωματωμένα συστήματα.

\section{Διάρθρωση της αναφοράς}
Η διάρθρωση της παρούσας διπλωματικής εργασίας είναι η εξής:

\begin{itemize}
    \item \textbf{Κεφάλαιο \ref{chapter:neuralNets}} Περιγράφονται τα σημαντικότερα νευρωνικά δίκτυα εντοπισμού αντικειμένων σε εικόνα. Γίνεται αναφορά στη δυνατότητα εκτέλεσής τους από ενσωματωμένες συσκευές και προς το σκοπό αυτό γίνονται και μετρήσεις.
    \item \textbf{Κεφάλαιο \ref{chapter:tl}} Δίνονται οι ορισμοί που αφορούν τη μετάδοση γνώσης. Αναφέρονται παραδείγματα και τεχνικές μετάδοσης γνώσης, όπως επίσης και το θεωρητικό υπόβαθρο για το μετέπειτα πειραματισμό. 
    \item \textbf{Κεφάλαιο \ref{chapter:experiments}} Παρουσιάζονται τα πειράματα από την περιγραφή τους, τα αποτελέσματά τους και τέλος τα συμπεράσματα που προέρχονται από αυτά.
    \item \textbf{Κεφάλαιο \ref{chapter:accel_SqueezeDet}} Παρουσιάζονται οι τρόποι και οι αλγόριθμοι που συντέλεσαν στην επιτάχυνση της εκπαίδευσης του νευρωνικού δικτύου \textit{SqueezeDet}.    
    \item \textbf{Κεφάλαιο \ref{chapter:architecture}} Επιγραμματική περιγραφή της υλοποίησης των πειραμάτων όσον αφορά το υλικό και το λογισμικό που χρησιμοποιήθηκε.
    \item \textbf{Κεφάλαιο \ref{chapter:futures}} Αναφέρονται τα προβλήματα που προέκυψαν και προτείνονται
    θέματα για μελλοντική μελέτη, αλλαγές και επεκτάσεις.
\end{itemize}
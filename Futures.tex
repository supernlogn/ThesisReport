\chapter{Μελλοντικές επεκτάσεις}
\label{chapter:futures}
Είναι σημαντικό να ερευνηθεί η εκτέλεση του πειράματος της ενότητας \ref{section:catastrophicForgetting}, ώστε να μπορέσει να εξετασθεί αν έχει νόημα η δημιουργία του γράφου στο Σχήμα \ref{fig:transferability_graph}, όπως επίσης και συστήματα που λειτουργούν κατά αυτό τον τρόπο. Όπως υποδεικνύει η βιολογία του εγκεφάλου τέτοια συστήματα εξυπηρετούν κατά πολύ στην χρήση ελάχιστων νευρώνων και την γρήγορη μάθηση καινούριων αντικειμένων. Όπως επίσης έχει υποδείξει και ο Hinton \cite{79} ο αριθμός των νευρώνων των νευρωνικών δικτύων επεξεργασίας εικόνας είναι κατά πολύ μεγαλύτερος από ότι χρειάζεται, όπως επίσης δεν χρειάζεται ενεργοποίηση όλων των νευρώνων για κάθε αντικείμενο. Για αυτό κρίνεται απαραίτητη η εξέταση της πλαστικότητας των νευρώνων ως προς την επανεκπαίδευση στις αρχιτεκτονικές νευρωνικών δικτύων με μικρό αριθμό νευρώνων.

Τα συστήματα που προτείνονται προς μελλοντική εξέταση είναι το MobileNet \cite{77} με το SSD, μία αρχιτεκτονική η οποία αναπτύχθηκε κατά τη διάρκεια εκπόνησης αυτής της εργασίας. Όπως επίσης, κρίνεται σκόπιμο να διερευνηθεί αν τα υπόλοιπα δίκτυα που σχετίζονται με το SqueezeDet, όπως το SqueezeDet+, VGG16 + ConvDet και ResNet50 + ConvDet, τα οποία αναφέρονται στο \cite{1}, έχουν την ίδια συμπεριφορά. Επίσης, επειδή ένας άλλος μεγάλος ανταγωνιστής του SqueezeDet είναι το YOLO (εξετάστηκε στο κεφάλαιο \ref{chapter:neuralNets}), προτείνεται και η διερεύνηση της μεταφαρσιμότητας και της πλαστικότητας των νευρώνων και αυτού του δικτύου.

Επιπλέον, λόγω πολλών ωρών εκτέλεσης της εκπαίδευσης δεν μπόρεσε να γίνει στα πλαίσια της διαθέσιμης τεχνολογίας και των εβδομάδων που διατίθενται σε μία διπλωματική εργασία η πλήρης εξερεύνηση της επίδοσης του αλγορίθμου στο χώρο των υπερπαραμέτρων. Ένα ολοκληρωμένων πείραμα με τη χρήση μόνο μίας εκ των μεθόδων υπολογίστηκε πως απαιτεί στην υπολογιστική συστοιχία του εργαστηρίου (\ref{section:hardware}) ~80 εβδομάδες. Οπότε, όταν βρεθούν οι κατάλληλοι πόροι για ικανό χρόνο θα ήταν σημαντική εξέταση για να δείξει αν υπάρχει βελτίωση/διατήρηση της συμπεριφοράς όχι μόνο στο παρόν πείραμα με το SqueezeDet αλλά και σε αυτό που βασίστηκε \cite{55}.

Αυτή η διπλωματική ξεκίνησε με σκοπό την αυτοματοποίηση της μετάδοσης γνώσης σε αλγορίθμους εντοπισμού αντικειμένων πραγματικού χρόνου και τη γρήγορη μετεκπαίδευση σε καινούρια σύνολα δεδομένων για την αποστολή παραμέτρων εντοπισμού αντικειμένων. Ωστόσο, προκειμένου να είναι εφικτός ο αυτοματισμός θα έπρεπε πρώτα να είναι γνωστή η συμπεριφορά του αλγορίθμου που θα χρησιμοποιόταν σε αυτό το σύστημα ως προς τη μεταφερσιμότητα των παραμέτρων του. Πλέον, αυτή είναι γνωστή για το SqueezeDet, οπότε και είναι δυνατή η κατασκευή αυτοματοποιημένου συστήματος μετάδοσης γνώσης βασιζόμενο σε αυτό. 

Μία ακόμα ενδιαφέρουσα προοπτική είναι η ανάπτυξη αυτού του συστήματος στο \textit{cloud} με το οποίο θα μπορούσαν να επικοινωνούν απευθείας όλες οι ενσωματωμένες συσκευές. Σε μία τέτοια προσπάθεια αν επιπλέον σχεδιάζεται να ενταχθούν περισσότερα δίκτυα/αλγόριθμοι, θα ήταν θεμιτή η ένταξη των αλγορίθμων σε ένα σύστημα όπως το Tensorflow Object Detection API\footnote{\url{https://github.com/tensorflow/models/tree/master/research/object_detection}} της \textit{Google}, το οποίο σκοπεύει στην υλοποίηση του ίδιου ακριβώς συστήματος, ωστόσο με αυτοματοποιημένο μόνο το κομμάτι της εκπαίδευση και όχι της επιλογής των υπερπαραμέτρων. Επίσης, μπορούν να ληφθούν υπόψη οι χρόνοι από την εκτέλεση του πειράματος στην ενότητα \ref{section:grantExpRes} για καλύτερη διαχείρηση του συστήματος.
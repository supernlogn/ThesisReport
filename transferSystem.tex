\chapter{Μηχανισμός μεταφοράς παραμέτρων}


\section{Περιληπτική ανάλυση του μηχανισμού}
Ο μηχανισμός που υλοποιήθηκε για αυτή την εργασία αποτελείται από δύο σκέλη. Το πρώτο αφορά την υπηρεσία που βρίσκεται στον server και αφορά την εκπαίδευση και επανεκπαίδευση νευρωνικών δικτύων. Στη μεριά του server όπως είναι φυσικό, το σύστημα έχει περισσότερους υπολογιστικούς πόρους, οπότε και το λογισμικό μπορεί να γραφεί σε πιο υψηλό επίπεδο. Το δεύτερο σκέλος αφορά το λογισμικό από πλευράς του ρομποτικού πράκτορα. Όπως είναι κατανοητό αυτό βρίσκεται σε ενσωματωμένη συσκευή και δεν έχει πολλούς υπολογιστικούς πόρους. Για αυτό και προσπαθούμε να μεταφέρουμε τον φόρτο εργασίας όσο περισσότερο γίνεται στην μεριά του server. Βέβαια να τονισθεί σε αυτό το σημείο πως αν οι ρομποτικοί πράκτορες γίνουν περισσότεροι και αυξηθούν οι υπολογιστικοί πόροι τους, τότε θα πρέπει να υπάρξει και μία συνεργασία, ώστε να ελαφρυνθεί υπολογιστικά ο server.

Στο κομμάτι του server ο χρήστης είναι ένας ρομποτικός πράκτορα για αυτό και μέσα στα διαγράμματα σημειώνεται ως \textit{robot}. Βέβαια αυτός ο χρήστης μπορεί να είναι και η εφαρμογή στο κινητό τηλέφωνο ενός ανθρώπου. Ωστόσο, για να διατηρηθεί η παρουσίαση της γενικότητας του μηχανισμού υποθέτουμε ένα πιο αφηρημένο είδος χρήστη.

Αντίστοιχα, στο κομμάτι του ρομποτικού πράκτορα / ενσωματωμένης συσκευής ο server μετ-εκπαίδευσης των βαρών θεωρείται ως εξωτερικό σύστημα. Αυτό επιτρέπει την ανάλυση και την υλοποίηση του συστήματος σε δύο ανεξάρτητα μέρη.

\section{Απαιτήσεις μηχανισμού μεταφοράς παραμέτρων}
Ο μηχανισμός μεταφοράς παραμέτρων στο στάδιο αυτής της εργασίας έχει λίγες απαιτήσεις σε αριθμό, διότι η κάθε μία από αυτές απαιτεί χρόνο όσον αφορά την ανάπτυξη της λύσης. Οι απαιτήσεις αφορούν μόνο τη μεταφορά παραμέτρων του νευρωνικού δικτύου SqueezeDet+ από τον server(σύστημα) προς τον ρομποτικό πράκτορα.

\begin{enumerate}
    \item Ο ρομποτικός πράκτορας πρέπει να μπορεί να αναζητήσει παραμέτρους εντοπισμού ενός σετ αντικειμένων που υπάρχουν στο σύστημα.
    \item Ο ρομποτικός πράκτορας πρέπει να μπορεί να απαιτήσεις τη δημιουργία παραμέτρων εντοπισμού ενός σετ αντικειμένων από το σύστημα.
    \item Ο ρομποτικός πράκτορας πρέπει να μπορεί να λάβει παραμέτρους εντοπισμού ενός σετ αντικειμένων που υπάρχουν στον server.
    \item Ο ρομποτικός πράκτορας πρέπει να μπορεί να ενημερώνεται για την κατάσταση στην οποία βρίσκεται η εκπαίδευση παραμέτρων ενός σετ αντικειμένων.
    \item Ο ρομποτικός πράκτορας πρέπει να μπορεί να λάβει την αρχιτεκτονική του νευρωνικού δικτύου εντοπισμού και τις απαραίτητες παραμέτρους για την κατασκευή του.
\end{enumerate}
 
Οι παραπάνω λειτουργικές απαιτήσεις αφορούν το βασικό στάδιο του λογισμικού και συνδέονται με τις παρακάτω μη λειτουργικές απαιτήσεις.
\begin{enumerate}
    \item Η επικοινωνία με τον ρομποτικό πράκτορα θα έπρεπε να γίνεται μέσω πρωτοκόλλου \textit{TCP/IP}.
    \item Η επικοινωνία με τον ρομποτικό πράκτορα θα έπρεπε να απαιτεί τη μεταφορά του μικρότερου δυνατού όγκου δεδομένων.
\end{enumerate}

Αυτές οι μη λειτουργικές απαιτήσεις ουσιαστικά θέτουν το πλαίσιο μέσα στο οποίο θα υλοποιηθεί η επικοινωνία και οι κλήσεις μεταξύ συστήματος και ρομποτικού πράκτορα. Συνολικά παρατηρείται ότι οι απαιτήσεις είναι ελάχιστες. Ωστόσο, δε θα μπορούσε να είναι περισσότερες διότι αυτές θέτουν τη βάση για μία ορθή και βάσιμη υλοποίηση που θα μπορεί να δεχθεί επέκταση μετέπειτα. Οποιεσδήποτε επιπλέον απαιτήσεις προστίθενται και στα αντίστοιχα κεφάλαια. Στο τέλος της εργασίας δίνεται μία πλήρη λίστα αυτών.

\section{Αρχιτεκτονική μηχανισμού μεταφοράς παραμέτρων}
\subsection{Εισαγωγικά}
Η αρχιτεκτονική του μηχανισμού μεταφοράς παραμέτρων ακολουθεί σαν βάση την αρχιτεκτονική που χρησιμοποιείται από ερευνητές για την δημιουργία καινούριων νευρωνικών δικτύων και για τον ολικό έλεγχό τους. Η προσέγγιση και το σκεπτικό που ακολουθήθηκε στην σχεδίαση ήταν ο διαχωρισμός των κομματιών που πράττουν διαφορετικό έργο κατά το δυνατό περισσότερο. Επίσης, όλες οι παράμετροι που χρησιμοποιούνται στο σύστημα να δίνονται ως ξεχωριστή είσοδος από αρχεία και σε αυτές τις παραμέτρους να ξεχωρίζουν αυτές που είναι \textbf{α}) μεταβαλλόμενες από το σύστημα \textbf{β}) μεταβαλλόμενες από τον ρομποτικό πράκτορα \textbf{γ}) σταθερές.

Επίσης χρησιμοποιείται και μία αρχιτεκτονική αρχείων για την αποθήκευση των παραμέτρων για εντοπισμό κάθε σετ αντικειμένων. Αν και θα μπορούσε να χρησιμοποιηθεί μία βάση δεδομένων, προτιμήθηκε να χρησιμοποιηθεί ένα αρχείο ευρετηρίου και το σύστημα αρχείων που ήδη έχει το εκάστοτε λειτουργικό, για καλύτερη συμβατότητα με τα σετ εκπαίδευσης.

\subsection{Τεχνολογίες και πακέτα χρήσης}
Το βασικότερο που πρέπει να τονιστεί σε αυτό το σημείο είναι πως η πλατφόρμα από μεριάς του server βασίστηκε σε λειτουργικό Linux Ubuntu 14.04, δεν ελέγχθηκε η λειτουργία σε κάποιο άλλο λειτουργικό. Επίσης η αρχιτεκτονική έχει αυτή τη μορφή εκτός από το συνολικό σκεπτικό και λόγω τριών αποφάσεων
\begin{itemize}
    \item της γλώσσας υλοποίησης που επιλέχθηκε (Python 2.7),
    \item της αρχιτεκτονικής της βιβλιοθήκης δημιουργίας νευρωνικών δικτύων TensorFlow (1.7)[???]
    \item της μορφής του σετ εκπαίδευσης ενός από τα pascal\_voc, ms_coco, google openImages.
    \item και της αρχιτεκτονικής REST εξυπηρέτησης αιτημάτων του ρομποτικού πράκτορα.
\end{itemize}



Η βιβλιοθήκη TensorFlow[???]:

Το σετ δεδομένων pascal\_voc[???]

\definecolor{mygreenii}{RGB}{124,166,198}
\definecolor{mygreeni}{RGB}{110,144,169}

\begin{forest}
  for tree={
    font=\sffamily,
    text=white,
    text width=2cm,
    minimum height=0.75cm,
    if level=0
      {fill=mygreenii}
      {fill=mygreeni},
    rounded corners=4pt,
    grow'=0,
    child anchor=west,
    parent anchor=south,
    anchor=west,
    calign=first,
    edge={mygreenii,rounded corners,line width=1pt},
    edge path={
      \noexpand\path [draw, \forestoption{edge}]
      (!u.south west) +(7.5pt,0) |- (.child anchor)\forestoption{edge label};
    },
    before typesetting nodes={
      if n=1
        {insert before={[,phantom]}}
        {}
    },
    fit=band,
    s sep=15pt,
    before computing xy={l=15pt},
  }
[text1
  [text1.1
    [text1.1.1]
    [text1.1.2]
    [text1.1.3]
  ]
  [text1.2
    [text1.2.1]
    [text1.2.2]
  ]
]
\end{forest}

Η αρχιτεκτονική REST[???, ???]



\subsection{Διάγραμμα πακέτων και κλάσεων}



\section{Υλοποίηση σε πλατφόρμες}

\subsection{Βελτιστοποιήσεις}

\section{Πειραματικά αποτελέσματα}